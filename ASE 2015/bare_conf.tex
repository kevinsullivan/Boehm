\documentclass[conference]{IEEEtran}

\hyphenation{op-tical net-works semi-conduc-tor}

\begin{document}
%
% paper title
% can use linebreaks \\ within to get better formatting as desired
\title{An Artificial Science for System Value Engineering and Assurance}

% author names and affiliations
% use a multiple column layout for up to three different
% affiliations
\author{\IEEEauthorblockN{Chong Tang}
\IEEEauthorblockA{Department of Computer Science\\
University of Virginia\\
Charlottesville, Virginia 22904-4740\\
Email: ctang@virginia.edu}
\and
\IEEEauthorblockN{Kevin Sullivan}
\IEEEauthorblockA{Department of Computer Science\\
University of Virginia\\
Charlottesville, Virginia 22904-4740\\
Email: sullivan@virginia.edu}
\and
\IEEEauthorblockN{Ke Dou}
\IEEEauthorblockA{Department of Computer Science\\
University of Virginia\\
Charlottesville, Virginia 22904-4740\\
Email: kdou@virginia.edu}
\and
\IEEEauthorblockN{Koleman Nix}
\IEEEauthorblockA{Department of Computer Science\\
University of Virginia\\
Charlottesville, Virginia 22904-4740\\
Email: jkn3wn@virginia.edu}
\and
\IEEEauthorblockN{Barry Boehm}
\IEEEauthorblockA{Department of Computer Science\\
University of Southern California\\
Los Angeles, California 90089-0781\\
Email: boehm@sunset.usc.edu}
}


% make the title area
\maketitle
\begin{abstract}
Research community of system quality attributes has been trying to define system qualities for a decade, but there is still no shared understanding for quality attributes, and yet we don't know how to assure them effectively. There are some theories and graphical tools like ACE and GSN, to help engineers to assure qualities in real projects. However, most graphical tools are not easy to scale. Besides, it's very hard to obtain feedback from engineers. Without the engineers' validation of such theories, it is very hard to evolve them if there are deficiencies exist. In this paper, we present a scientific theory and a framework for quality definition and assurance. We formalize a top-down taxonomy for quality attributes, and then integrates it with our previous work, a bottom-up formal language for quality requirements analysis, for quality attributes definition and assurance. Our theory uses the semantic formal language to parse the quality requirements to ilities, and provides evidence as the leafs of the top-down taxonomy. It then assures the the highest level quality by deductively reasoning the lower levels of qualities originally start from the leafs. We also present an illustration example by using this framework to conduct quality assurance analysis of a smart home project.
\end{abstract}

\IEEEpeerreviewmaketitle

\section{Introduction}
The value of system arises from a combination of many system qualities: from functionality and dependability to usability, flexibility, and others. Today our understanding of and ability to engineer system qualities is weak. There are no consensus definitions for many qualities. We have a poor understanding of means-ends relationships among qualities. We lack frameworks for reasoning about tradeoffs. Our ability to manage them across lifecycle is weak. These engineering problems are in turn rooted in weak science. Research in this area has often been informal and imprecise, with theories expressed in natural language, tables, graphics, and without the benefit of the kinds of mathematical, computational, and logical notations that are critical to producing clear, unambiguous, testable, generalized theories. Consequently we also lack foundations for automated tools for assisting with such issues, which are in turn a key to the dissemination, testing, validation, and eventual adoption and use of fundamental concepts of comprehensive quality requirements and engineering. We offer an approach that combines Bosch?s concept of innovation experiment systems with the use of rigorous formal specification and software synthesis using constructive logic proof assistant technology. We refine and express quality theories using the highly expressive language of the Coq proof assistant, which is capable of unified treatment of mathematical, computational, and logical concerns. We use synthesis from such specifications to support continuous deployment of web-based software implementations of the concepts embodied in such theories, and  user-driven testing based on such tools to drive theory testing, evolution, and validation.  Elements of the specific framework that we are constructing include a hierarchy of qualities and relationships parameterized by stakeholders, contexts, and system operational stages, based on recent work by Boehm; quality-specific specification languages for expressing detailed requirements (based on recent work by Ross and Rhodes); and a novel integration of the distinct, previously conflicting theories underlying these two efforts. We present our overall approach and illustrate its application with an example system for home automation. The overall contribution of this work is a novel, rigorous, and promising new approach to developing, promulgating, testing, evolving, and validating the scientific theory that is needed to underpin rigorous new approaches to comprehensive system quality engineering.

\section{Related Work}
Significant progress in these areas has been made for certain properties, particularly for system dependability properties, but not for others, such as flexibility, security, resilience, and many others. Yet the challenge in systems engineering is to produce value by achieving quality across broad ranges of system properties.
\begin{itemize}
	\item Our overall approach is to create a scientific innovation experiment system to develop an integrated framework for quality definition, specification, realization, assurance
	\begin{itemize} 
		\item formal theory expressed using highly expressive language capable of unified treatment of mathematical, computational, and logical concerns that arise in relation to definition, specification, and assurances of diverse system qualities
		\item	continuous deployment supported by software synthesis from such expressions to create tools for concept dissemination, user-driven validation, and evolution
	\end{itemize}
	\item Elements of the specific framework that we are constructing
	\begin{itemize}
		\item hierarchies of qualities and relationships (Boehm)
		\item languages for expressing specific requirements (e.g., Ross)
		\item formalization of these preceding constructs 
		\item integrate diverse approaches to system qualities, e.g., Boehm?s top-down system quality taxonomy and Ross?s bottom-up semantic approach
		\item reusable theories that can be instantiated for many projects (parameterization)
	\end{itemize}
\end{itemize}

\subsection{Shortcoming of Related Work}
\begin{itemize}
\item Weak engineering foundations
	\begin{itemize}
		\item No consensus definitions for many system properties
		\item Weak understanding of means-ends relationships among system properties
		\item Lack a framework for reasoning about tradeoffs among system qualities
		\item Poor ability to manage full range of system qualities across system lifecycle 
		\item	Fundamental weaknesses in our ability to handle system requirements 
	\end{itemize}
	\item	Weak scientific foundations
	\begin{itemize}
		\item Lack of rigor in research, notations, and methods employed in this area
		\begin{itemize}
			\item Research in this area has been informal, qualitative, even sloppy 
			\item	Theories are expressed informally: natural language, tables, graphics
			\item	No use of mathematical, computational, and logical notations critical to producing clear, unambiguous, testable theories
			\item Weak basis for automation, a key to dissemination, testing, validation, and to adoption and evolution of basic concepts, method, and tools
			\item Lack of rigorous and usable notations / languages for specifying the full range of system qualities that systems have to have
		\end{itemize}
		\item	Weak understanding the nature of assurance, in particular in understanding the role and interpretation of evidence and the relationship between inductive and deductive reasoning in system quality assurance
	\end{itemize}
\end{itemize}

\section{Approach}
\begin{itemize}
	\item	Our overall approach is to create a scientific innovation experiment system to develop an integrated framework for quality definition, specification, realization, assurance 
	\begin{itemize}
		\item	formal theory expressed using highly expressive language capable of unified treatment of mathematical, computational, and logical concerns that arise in relation to definition, specification, and assurances of diverse system qualities
		\item	continuous deployment supported by software synthesis from such expressions to create tools for concept dissemination, user-driven validation, and evolution
	\end{itemize}
	\item	Elements of the specific framework that we are constructing
	\begin{itemize}
		\item	hierarchies of qualities and relationships (Boehm)
		\item	languages for expressing specific requirements (e.g., Ross)
		\item	formalization of these preceding constructs 
		\item	integrate diverse approaches to system qualities, e.g., Boehm?s top-down system quality taxonomy and Ross?s bottom-up semantic approach
		\item	reusable theories that can be instantiated for many projects (parameterization)
	\end{itemize}
\end{itemize}

\section{Evaluation}
We also present an illustration example by using this framework to conduct quality assurance analysis of a smart home project.

\section{Conclusion}
The conclusion goes here.


% use section* for acknowledgement
\section*{Acknowledgment}
The authors would like to thank...

\bibliography{mybib}{}
\bibliographystyle{plain}

% that's all folks
\end{document}


